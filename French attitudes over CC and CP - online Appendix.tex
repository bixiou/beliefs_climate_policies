\documentclass[11pt]{article}
\usepackage[utf8]{inputenc}
\usepackage[T1]{fontenc}
\usepackage{hyperref}
\usepackage[authoryear]{natbib}
\usepackage{textcomp}
\usepackage{subcaption}
\usepackage{graphicx}
\usepackage{fancybox}
\usepackage{xcolor}
\usepackage{makeidx}
\usepackage{float}
\usepackage{amsmath,amssymb}
\usepackage{eurosym}

\usepackage{enumitem}
\PassOptionsToPackage{normalem}{ulem}
\usepackage{ulem}

\usepackage{multicol}
\usepackage[toc,page]{appendix}
\usepackage{array,multirow,makecell}
\usepackage[modulo]{lineno}
\renewcommand{\arraystretch}{0.73}
\setcellgapes{1pt}
\makegapedcells
\renewcommand*\thetable{\Roman{table}}
\renewcommand*\thefigure{\Roman{figure}}
\renewcommand{\thetable}{\Alph{section}.\arabic{table}}
\renewcommand{\thefigure}{\Alph{section}.\arabic{figure}}
\usepackage{chngcntr}
\counterwithin{figure}{section}
\counterwithin{table}{section}
\renewcommand{\floatpagefraction}{1}
%\renewcommand{\footnotelayout}{\setstretch{0.5}}
\newcolumntype{R}[1]{>{\raggedleft\arraybackslash }b{#1}}
\newcolumntype{L}[1]{>{\raggedright\arraybackslash }b{#1}}
\newcolumntype{C}[1]{>{\centering\arraybackslash }b{#1}}
\usepackage[left=2.5cm,right=2.5cm,top=2.5cm,bottom=2.5cm]{geometry}
\linespread{1.5}
\date{}
\hypersetup{citecolor=blue,colorlinks=true}
%\setlength{\parindent}{1pt}

\title{Supplementary material - French Attitudes over Climate Change and Climate Policies}
\author{Thomas Douenne and Adrien Fabre\footnote{Douenne: Paris School of Economics, Université Paris 1 Panthéon-Sorbonne, 48 Boulevard Jourdan, 75014, Paris, France (email: thomas.douenne@psemail.eu); Fabre: Paris School of Economics, Université Paris 1 Panthéon-Sorbonne, 48 Boulevard Jourdan, 75014, Paris, France (email: adrien.fabre@psemail.eu)}} 
\date{June 2019}

\linespread{1}\selectfont

\begin{document}
\maketitle
%\sloppy


  
%\emph{This online appendix contains supplementary material to the paper titled ``French Attitudes over Climate Change and Climate Policies''. In a first section we detail the survey.}






%\setcounter{tocdepth}{2}
%\tableofcontents
%%TC:endignore

%\vfill\eject 
%\linenumbers

% \section{Survey}

%     \subsection{Questionnaire\label{app:questionnaire}}

% Hereafter, we only describe questions of the survey that are used
% in the present paper. The other questions are described and analyzed
% in our companion paper \citep{douenne_can_2019}.

% \paragraph{Socio-demographics}
% \begin{enumerate}[resume,leftmargin=*]
% \item What is your postal code? 
% \item What is your gender (in the sense of civil status)? \emph{}\\
% \emph{Female; Male }
% \item What is your age group? \emph{}\\
% \emph{18 to 24 years old; 25 to 34 years old; 35 to 49 years old;
% 50 to 64 years old; 65 years old or more} 
% \item What is your employment status? \emph{}\\
% \emph{Permanent; Temporary contract; Unemployed; Student; Retired;
% Other active; Inactive}
% \item What is your socio-professional category? (Remember that the unemployed
% are active workers). \emph{}\\
% \emph{Farmer; Craftsperson, merchant; Independent; Executive; Intermediate
% occupation; Employee; Worker; Retired; Other Inactive} 
% \item What is your highest degree? \emph{}\\
% \emph{No diploma; Brevet des collèges; CAP or BEP {[}secondary{]};
% Baccalaureate; Bac +2 (BTS, DUT, DEUG, schools of health and social
% training...); Bac +3 (licence...) {[}bachelor{]}; Bac +5 or more (master,
% engineering or business school, doctorate, medicine, master, DEA,
% DESS...)}
% \item How many people live in your household? Household includes: you, your
% family members who live with you, and your dependents. 
% \item What is your net \textbf{\uline{monthly}} income (in euros)? \textbf{\uline{All
% income}} (before withholding tax) is included here: salaries, pensions,
% allowances, APL {[}housing allowance{]}, land income, etc. 
% \item What is the net \textbf{\uline{monthly}} income (in euros) \textbf{\uline{of
% your household}}? \textbf{\uline{All income}} (before withholding
% tax) is included here: salaries, pensions, allowances, APL {[}housing
% allowance{]}, land income, etc. 
% \item In your household how many people are 14 years old or older (\textbf{\uline{including
% yourself}})? 
% \item In your household, how many people are over the age of majority (\textbf{\uline{including
% yourself}})? 
% \end{enumerate}

% \paragraph{Energy characteristics}
% \begin{enumerate}[resume,leftmargin=*]
% \item What is the surface area of your home? (in m\texttwosuperior )
% \item What is the heating system in your home? \emph{}\\
% \emph{Individual heating; Collective heating; PNR (Don't know, don't
% say)}
% \item What is the main heating energy source in your home? \emph{}\\
% \emph{Electricity Town gas; Butane, propane, tank gas; Heating oil;
% Wood, solar, geothermal, aerothermal (heat pump); Other; PNR (Don't
% know, don't say)}
% \item How many motor vehicles does your household have? \emph{}\\
% \emph{None; One; Two or more} 
% \item {[}Without a vehicle{]} How many kilometers have you driven in the
% last 12 months? 
% \item {[}One vehicle{]} What type of fuel do you use for this vehicle? \emph{}\\
% \emph{Electric or hybrid; Diesel; Gasoline; Other} 
% \item {[}One vehicle{]} What is the average fuel economy of your vehicle?
% (in Liters per 100 km)
% \item {[}One vehicle{]} How many kilometers have you driven with your vehicle
% in the last 12 months?
% \item {[}At least two vehicles{]} What type of fuel do you use for your
% main vehicle?\\
%  \emph{Electric or hybrid; Diesel; Gasoline; Other} 
% \item {[}At least two vehicles{]} What type of fuel do you use for your
% second vehicle?\\
%  \emph{Electric or hybrid; Diesel; Gasoline; Other} 
% \item {[}At least two vehicles{]} What is the average fuel economy of all
% your vehicles? (in Liters per 100 km) 
% \item {[}At least two vehicles{]} How many kilometers have you driven with
% all your vehicles in the last 12 months? 
% \end{enumerate}

% \paragraph{Partial reforms {[}transport / housing{]}}

% (...)\emph{}
% \begin{enumerate}[resume,leftmargin=*]
% \item If fuel prices increased by 50 cents per liter, by how much would
% \textbf{\uline{your household}} reduce its fuel consumption? \emph{}\\
% \emph{0\% -} {[}\emph{I already consume almost none }/\emph{ I am
% already not consuming}{]}\emph{; 0\% - }{[}\emph{I am constrained
% on all my trips} / \emph{I will not reduce it}{]}\emph{; From 0\%
% to 10\%; From 10\% to 20\%; From 20\% to 30\%; More than 30\% - }{[}\emph{I
% would change my travel habits significantly }/ \emph{I would change
% my consumption significantly}{]}
% \item In your opinion, if {[}fuel prices increased by 50 cents per liter
% / gas and heating oil prices increased by 30\%{]}, by how much would
% \textbf{\uline{French people}} reduce their consumption on average?
% \emph{}\\
% \emph{From 0\% to 3\%; From 3\% to 10\%; From 3\% to 10\%; From 10\%
% to 20\%; From 20\% to 30\%; More than 30\%} 
% \end{enumerate}

% \paragraph{Tax \& dividend: initial}
% \begin{enumerate}[resume,leftmargin=*]
% \item The government is studying an increase in the carbon tax, whose revenues
% would be redistributed to all households, regardless of their income.
% This would imply: 
% \end{enumerate}
% \begin{itemize}
% \item an increase in the price of gasoline by 11 cents per liter and diesel
% by 13 cents per liter; 
% \item an increase of 13\% in the price of gas, and 15\% in the price of
% heating oil;
% \item an annual payment of 110\euro{} to each adult, or 220\euro{} per year for a couple.
% \\
% \\
% (...)
% \end{itemize}
% \begin{enumerate}[resume,leftmargin=*]
% \item {[} {[}empty{]} / Scientists agree that a carbon tax would be effective
% in reducing pollution.{]} Do you think that such a measure would reduce
% pollution and fight climate change? \emph{}\\
% \emph{Yes; No; PNR (Don't know, don't say)}
% \item In your opinion, which categories would lose {[} {[}blank{]} / purchasing
% power{]} with such a measure? (Several answers possible) \emph{}\\
% \emph{No one; The poorest; The middle classes; The richest; All French
% people; Rural or peri-urban people; Some French people, but not a
% particular income category; PNR (Don't know, don't say)} 
% \item In your opinion, what categories would gain purchasing power with
% such a measure? (Several answers possible) \emph{}\\
% \emph{No one; The poorest; The middle classes; The richest; All French
% people; Urban dwellers; Some French people, but not a particular income
% category; PNR (Don't know, don't say)} 
% \end{enumerate}

% \paragraph{Tax \& dividend: after knowledge}

% We always consider the same measure. (...)
% \begin{enumerate}[resume,leftmargin=*]
% \item Why do you think this measure is beneficial? (Maximum three responses)
% \emph{}\\
% \emph{Contributes to the fight climate change; Reduces the harmful
% effects of pollution on health; Reduces traffic congestion; Increases
% my purchasing power; Increases the purchasing power of the poorest;
% Fosters France's independence from fossil energy imports; Prepares
% the economy for tomorrow's challenges; For none of these reasons;
% Other (specify): }
% \item Why do you think this measure is unwanted? (Maximum three answers)
% \emph{}\\
% \emph{Is ineffective in reducing pollution; Alternatives are insufficient
% or too expensive; Penalizes rural areas; Decreases my purchasing power;
% Decreases the purchasing power of some modest households; Harms the
% economy and employment; Is a pretext for raising taxes; For none of
% these reasons; Other (specify):} 
% \end{enumerate}
% (...)

% \paragraph{Attitudes over other policies}
% \begin{enumerate}[resume,leftmargin=*]
% \item In which cases would you be in favor of increasing the carbon tax?
% I would be in favor if the tax revenues were used to finance...\emph{ }
% \begin{enumerate}[resume,leftmargin=*]
% \item a payment to the 50\% poorest French people (those earning less than
% 1670\euro{} per month) 
% \item a payment to all French people 
% \item a compensation for households forced to consume petroleum products
% \item a decrease in social contributions
% \item a decrease in VAT 
% \item a decrease in the public deficit 
% \item the thermal renovation of buildings 
% \item renewable energy (wind, solar, etc.)
% \item clean transport
% \end{enumerate}
% \end{enumerate}
% \emph{Yes, absolutely; Yes, rather; Indifferent or Don't know; No,
% not really; No, not at all}
% \begin{enumerate}[resume,leftmargin=*]
% \item Please select ``A little'' (test to check that you are attentive).
% \emph{}\\
% \emph{Not at all; A little; A lot; Completely; PNR (Don't know, don't
% say)} 
% \item Would you support the following environmental policies? 
% \begin{enumerate}[resume,leftmargin=*]
% \item A tax on kerosene (aviation) 
% \item A tax on red meat 
% \item Stricter standards on the insulation of new buildings 
% \item Stricter standards on the pollution of new vehicles
% \item Stricter standards on pollution during roadworthiness tests 
% \item The prohibition of polluting vehicles in city centers 
% \item The introduction of urban tolls 
% \item A contribution to a global climate fund 
% \end{enumerate}
% \end{enumerate}
% \emph{Yes, absolutely; Yes, rather; Indifferent or Don't know; No,
% not really; No, not at all}
% \begin{enumerate}[resume,leftmargin=*]
% \item For historical reasons, diesel is taxed less than gasoline. Would
% you be in favor of raising taxes on diesel to catch up with the level
% of taxation on gasoline? \emph{}\\
% \emph{Yes; No; PNR (Don't know, don't say) }
% \end{enumerate}

% \paragraph{Attitudes over climate change}
% \begin{enumerate}[resume,leftmargin=*]
% \item How often do you talk about climate change? \emph{}\\
% \emph{Several times a month; Several times a year; Almost never; PNR
% (Don't know, don't say) }
% \item In your opinion, climate change... \emph{}\\
% \emph{is not a reality; is mainly due to natural climate variability;
% is mainly due to human activity; PNR (Don't know, don't say). }
% \item Which of the following elements contribute to global warming? (Several
% answers possible) \emph{}\\
% \emph{CO$_{2}$; Methane; Oxygen; Particulate matter}
% \item In your opinion, which of the following statements are true? (Several
% answers possible). \emph{}\\
% \emph{Consuming one beef steak emits about 20 times more greenhouse
% gases than eating two servings of pasta.; Electricity produced by
% nuclear power emits about 20 times more greenhouse gases than electricity
% produced by wind turbines.; A seat in a Bordeaux - Nice journey emits
% about 20 times more greenhouse gases by plane than by high speed train. }
% \item In your opinion, how would the effects of climate change be, if humanity
% did nothing to limit it? \emph{}\\
% \emph{Insignificant, or even beneficial; Small, because humans would
% be able to live with it; Grave, because there would be more natural
% disasters; Disastrous, lifestyles would be largely altered; Cataclysmic,
% humankind would disappear; PNR(Don't know, don't say) }
% \item In which of these two regions do you think will climate change have
% the worst consequences? \emph{}\\
% \emph{The European Union; India; As much in both }
% \item In your opinion, in France, which generations will be seriously affected
% by climate change? (Several answers possible) \emph{}\\
% \emph{People born in the 1960s; People born in the 1990s; People born
% in the 2020s; People born in the 2050s; None of the four }
% \item In your opinion, who is responsible for climate change? (Several possible
% choices) \emph{}\\
% \emph{Each of us; The richest; Governments; Some foreign countries;
% Past generations; Natural causes }
% \item Currently, each French person emits on average the equivalent of 10
% tons of CO$_{2}$ per year. \\
% \\
% In your opinion, how much must this figure be reduced to by 2050 in
% order to hope to contain global warming to +2°C in 2100 (if all countries
% did the same)? In 2050, we should emit at most... \emph{}\\
% \emph{0; 1; 2; 3; 4; 5; 6; 7; 8; 9; 10} tons 
% \item Has climate change had or will it have an influence on your decision
% to make a child (or children)?\emph{ }\\
% \emph{Yes; No; PNR (Don't know, don't say)}
% \item {[}If \emph{Yes}{]} Why does climate change influence your decision
% to have a child (or children)? (Several answers possible). \emph{}\\
% \emph{Because I don't want my child to live in a devastated world.;
% Because each additional human being aggravates climate change.}
% \item Would you be willing to change your lifestyle to fight climate change?
% (Several answers possible) \emph{}\\
% \emph{Yes, if policies went in this direction; Yes, if I had the financial
% means; Yes, if everyone did the same; No, only the richest people
% have to change their way of life; No, it is against my personal interest;
% No, I think climate change is not a real problem; I have already adopted
% a sustainable way of life; I try, but I have trouble changing my habits} 
% \item Assuming that all states in the world agree to firmly fight climate
% change, notably through a transition to renewable energy, by making the richest contribute, and imagining that France would expand the
% supply of non-polluting transport very widely; would you be willing
% to adopt an ecological lifestyle (i.e. eat little red meat and ensure
% to use almost no gasoline, diesel or kerosene)? \emph{}\\
% \emph{Yes; No; PNR (Don't know, don't say) }
% \end{enumerate}

% \paragraph{Shale gas (and smoking)}
% \begin{enumerate}[resume,leftmargin=*]
% \item Do you smoke regularly? \emph{Yes; No }
% \item The use of shale gas would limit climate change, as gas would be exported
% and used to produce electricity instead of coal. On the other hand,
% extraction would risk reducing water quality at the local level. Your
% department {[}would possibly be / would not be{]} concerned by the exploitation
% of shale gas. \\
% \\
% In view of this information, would you be in favor of shale gas exploitation
% in France? \emph{}\\
% \emph{Yes; No; PNR (Don't know, don't say) }
% \item What would be the main benefit to you from shale gas development?
% \emph{}\\
% \emph{This would limit climate change; This would create jobs and
% boost the department; None of these two reasons }
% \item What do you think of the idea that shale gas would limit climate change?
% \emph{}\\
% \emph{It is valid: any decrease in emissions goes in the right direction;
% It is unwelcome: emissions should be stopped, not just slowed down;
% PNR(Don't know, don't say) }
% \end{enumerate}

% \paragraph{Access to public transport and mobility habits}
% \begin{enumerate}[resume,leftmargin=*]
% \item How many minutes walk is it to the nearest public transit stop? (To
% simplify, you can use the conversion 1 km = 10 min walk). \emph{}\\
% \emph{in min:} ; \emph{PNR (Don't know, don't say) }
% \item How often does the nearest public transport pass? (excluding school
% buses) \emph{}\\
% \emph{Less than three times a day; Between four times a day and once
% an hour; Once or twice an hour; More than three times an hour; PNR
% (Don't know, don't say) }
% \item What do you think about the availability of public transport where
% you live? It is... \emph{}\\
% \emph{Satisfactory; Suitable, but should be increased; Limited, but
% sufficient; Insufficient; PNR (Don't know, don't say) }
% \item What mode of transportation do you mainly use for each of the following
% trips?
% \begin{enumerate}[resume,leftmargin=*]
% \item Home - work (or studies) 
% \item Grocery shopping 
% \item Leisure (excluding holidays) 
% \end{enumerate}
% \end{enumerate}
% \emph{Car; Public transport; Walking or cycling; Two-wheeled vehicle;
% Carpooling;} \emph{Not concerned} 
% \begin{enumerate}[resume,leftmargin=*]
% \item {[}If \emph{Car }selected for Work{]} Would it be possible for you,
% without changing your home or workplace, to travel from home to work
% using public transport? \emph{}\\
% \emph{Yes, it would not be very difficult for me; Yes, but it would
% bother me; No; PNR (Don't know, don't say) }
% \item {[}If \emph{Car }selected for Work{]} Would it be possible for you,
% without changing your home or workplace, to travel from home to work
% by walking or cycling? \emph{}\\
% \emph{Yes, it would not be very difficult for me; Yes, but it would
% bother me; No; PNR (Don't know, don't say) }
% \end{enumerate}

% \paragraph{Politics and media}
% \begin{enumerate}[resume,leftmargin=*]
% \item How much are you interested in politics? \emph{}\\
% \emph{Almost not; A little; A lot }
% \item How would you define yourself? (Several answers possible) \emph{}\\
% \emph{Extreme left; Left; Center; Right; Extreme right; Liberal; Conservative;
% Humanist; Patriot; Apolitical; Ecologist }
% \item How do you keep yourself informed of current events? Mainly through...
% \emph{}\\
% \emph{Television; Press (written or online); Social networks; Radio;
% Other}
% \item What do you think of the Yellow Vests? (Several answers possible)
% \emph{}\\
% \emph{I am part of them; I support them; I understand them; I oppose
% them; PNR (Don't know, don't say) }
% \end{enumerate}

% \paragraph{Open field}
% \begin{enumerate}[resume,leftmargin=*]
% \item The survey is nearing completion. You can now enter any comments,
% comments or suggestions in the field below.
% \end{enumerate}

%     \subsection{Wrong and correct answers: sources}

%         \subsubsection{Carbon footprints\label{app:footprint}}

% \paragraph{Plane vs. train}

% Given that French electricity mix is decarbonized at 93\%\footnote{Cf. \hyperlink{https://www.rte-france.com/sites/default/files/be_pdf_2018v3.pdf}{RTE - Bilan électrique 2018} (p. 32).}, the carbon footprint of highspeed train is actually more than 20 times lower than that of an interior flight of the same distance. Hence, we chose Bordeaux - Nice as our case study as the train connection makes a big detour by Paris. Thus, we obtain an emission of 10 kg of CO$_\textnormal{2}$ by train as compared to 180 kg by plane. Our source for train is the French railroad company, \hyperlink{https://www.oui.sncf/aide/calcul-des-emissions-de-co2-sur-votre-trajet-en-train}{SNCF}, and is consistent with data aggregated by the official agency \hyperlink{basecarbone.fr}{ADEME}. For the flight, our source is a  \hyperlink{https://calculator.carbonfootprint.com/calculator.aspx?tab=3}{carbon footprint calculator}. \hyperlink{http://www.climatecare.org/home.aspx}{Another calculator} provides almost the same result, so we preferred this figure rather than a higher figure from a \hyperlink{https://co2.myclimate.org/fr/flight_calculators}{third calculator}.

% \paragraph{Nuclear vs. wind}

% AR5 from \citet{ar5_ar5_nodate} and \citet{pehl_understanding_2017} show that nuclear power plants and wind turbines have similar carbon footprint, at 10 gCO$_\textnormal{2}$eqq$/$kWh (for comparison, it is 500 for gas combined cycle).

% \paragraph{Beef vs. pasta}

% \citet{poore_reducing_2018} show that median beef carbon footprint is 60 kgCO$_\textnormal{2}$eqq$/$kg (more precisely, 30 kgCO$_\textnormal{2}$eqq per 100g of protein and 200g of protein per kg); while the carbon footprint of wheat pasta is 1.3 kgCO$_\textnormal{2}$eqq$/$kg (0.5 kgCO$_\textnormal{2}$eqq per 1000 kcal of protein and 2695 kcal per kg). Given that a beef steak \hyperlink{http://www.lessentieldesviandes-pro.org/introduction.php}{weighs 100-125g}, its carbon footprint is twenty times that of two servings of pasta of 125g each. % DONE: au temps pour moi je m'étais embrouillé en refaisant les calculs.

%         \subsubsection{Current and target emissions\label{app:emission}}

% French consumption-based yearly GhG emissions amounted in 2014 to 712 MtCO$_\textnormal{2}$eqq, i.e. 10.8 tCO$_\textnormal{2}$eqq p.c., and are roughly stable in recent years \citep{cgdd_chiffres_2019}. To stop climate change and stabilize the GhG concentration in the atmosphere, it is required to meet zero net emissions. To meet the Paris agreement,  \hyperlink{https://www.ecologique-solidaire.gouv.fr/strategie-nationale-bas-carbone-snbc}{France National Low-Carbon Strategy} aims to achieve carbon (i.e. GhG) neutrality by 2050 \citep{ministry_of_ecology_france_2015}. Given carbon sinks estimated at 85 Mt$_\textnormal{2}$eqq for 2050 (mainly forest and soil), this strategy requires to reach gross emissions of about 1 tCO$_\textnormal{2}$eqq p.c. at this date. Admittedly, less stringent scenarios may still allow to keep global warming below +2\textdegree{}C in 2100 with good probability --- even considering the same burden share for France --- by relying more heavily on net negative emissions after 2070 through carbon capture and storage. For this reason, we consider a range of answers as correct for the French target emission in 2050: from 0 to 2 tCO$_\textnormal{2}$eqq p.c.


% \begin{multicols}{2}[\subsection{Raw data\label{sec:Raw-Data}}]

% \renewcommand{\arraystretch}{0.73}

% \begin{table}[H]
% \label{table:sample_characteristics}
% \caption{\label{tab:Sample-Characteristics}Sample characteristics: quotas stratas.}
% \centering
% \begin{tabular}{lcc}
% \hline \hline  \\[-1.8ex]
%  & \emph{Population} & Sample  \tabularnewline \\[-1.8ex]
% \hline  \\[-1.8ex]
% \textbf{gender} & & \tabularnewline  \\[-1.8ex]
% woman & \emph{0.52} & 0.53\tabularnewline
% man & \emph{0.48} & 0.47\tabularnewline
% \hline \\[-1.8ex]
% \textbf{age} &  & \tabularnewline  \\[-1.8ex]
% 18-24 & \emph{0.12} & 0.11\tabularnewline
% 25-34 & \emph{0.15} & 0.11\tabularnewline
% 35-49 & \emph{0.24} & 0.24\tabularnewline
% 50-64 & \emph{0.24} & 0.26\tabularnewline
% >65 & \emph{0.25} & 0.27\tabularnewline
% \hline \\[-1.8ex]
% \textbf{profession} &  & \tabularnewline  \\[-1.8ex]
% farmer & \emph{0.01} & 0.01\tabularnewline
% independent & \emph{0.03} & 0.04\tabularnewline
% executive & \emph{0.09} & 0.09\tabularnewline
% intermediate & \emph{0.14} & 0.14\tabularnewline
% employee & \emph{0.15} & 0.16\tabularnewline
% worker & \emph{0.12} & 0.13\tabularnewline
% retired & \emph{0.33} & 0.33\tabularnewline
% inactive & \emph{0.12} & 0.11\tabularnewline
% \hline  \\[-1.8ex]
% \textbf{education} &  & \tabularnewline  \\[-1.8ex]
% No diploma or \emph{Brevet} & \emph{0.30} & 0.24\tabularnewline
% \emph{CAP} or \emph{BEP} & \emph{0.25} & 0.26\tabularnewline
% \emph{Bac} & \emph{0.17} & 0.18\tabularnewline
% Higher & \emph{0.29} & 0.31\tabularnewline
% \hline  \\[-1.8ex]
% \textbf{size of town} &  & \tabularnewline  \\[-1.8ex]
% rural & \emph{0.22} & 0.24\tabularnewline
% <20k & \emph{0.17} & 0.18\tabularnewline
% 20-99k & \emph{0.14} & 0.13\tabularnewline
% >100k & \emph{0.31} & 0.29\tabularnewline
% Paris area & \emph{0.16} & 0.15\tabularnewline
% \hline  \\[-1.8ex]
% \textbf{region} &  & \tabularnewline  \\[-1.8ex]
% \emph{IDF} & \emph{0.19} & 0.17\tabularnewline
%  \emph{Nord} & \emph{0.09} & 0.10\tabularnewline
%  \emph{Est} & \emph{0.13} & 0.12\tabularnewline
% \emph{SO} & \emph{0.09} & 0.09\tabularnewline
% \emph{Centre} & \emph{0.10} & 0.12\tabularnewline
%  \emph{Ouest} & \emph{0.10} & 0.10\tabularnewline
%  \emph{Occ} & \emph{0.09} & 0.09\tabularnewline
% \emph{ARA} & \emph{0.12} & 0.13\tabularnewline
% \emph{PACA} & \emph{0.09} & 0.09\tabularnewline  \\[-1.8ex]
% \hline \hline 
% \end{tabular}\bigskip{}
% \end{table}


% \begin{table}[H]
%     \caption{Households' characteristics.\label{tab:app-energetic-characs}}
% \centering
% \begin{tabular}{lcc}
% \hline \hline  \\[-1.8ex]
%  & \emph{Population} & Sample  \tabularnewline \\[-1.8ex]
% \hline  \\[-1.8ex]
% \multicolumn{3}{l}{\textbf{Household composition (mean)}} \tabularnewline  \\[-1.8ex]
% Household size & \emph{2.36} & 2.38\tabularnewline
% Number of adults & \emph{2.03} & 1.93\tabularnewline
% c.u. & \emph{1.60} & 1.61\tabularnewline
% \hline   \\[-1.8ex]
% \multicolumn{3}{l}{\textbf{Energy source (share)}} \tabularnewline  \\[-1.8ex]
% Gas & \emph{0.42} & 0.36\tabularnewline
% Fuel & \emph{0.12} & 0.09\tabularnewline
% \hline   \\[-1.8ex]
% \multicolumn{3}{l}{\textbf{Accomodation surface (m$^\textnormal{2}$)}} \tabularnewline  \\[-1.8ex]
% mean & \emph{97} & 96\tabularnewline
% p25 & \emph{69} & 66\tabularnewline
% p50 & \emph{90} & 90\tabularnewline
% p75 & \emph{120} & 115\tabularnewline
% \hline   \\[-1.8ex]
% \multicolumn{3}{l}{\textbf{Distance traveled by car (km/year)}} \tabularnewline  \\[-1.8ex]
% mean & \emph{13,735} & 15,328\tabularnewline
% p25 & \emph{4,000} & 4,000\tabularnewline
% p50 & \emph{10,899} & 10,000 \tabularnewline
% p75 & \emph{20,000 } & 20,000 \tabularnewline
% \hline   \\[-1.8ex]
% \multicolumn{3}{l}{\textbf{Fuel economy (L/100 km)}} \tabularnewline  \\[-1.8ex]
% mean & \emph{6.39} & 7.25\tabularnewline
% p25 & \emph{6} & 5\tabularnewline
% p50 & \emph{6.5} & 6\tabularnewline
% p75 & \emph{7.5} & 7\tabularnewline  \\[-1.8ex]
% \hline \hline 
% \end{tabular}\bigskip{}

%     % \\ $\quad$ \\
%      \footnotesize{\textsc{Sources:} Matched BdF; except for number of adults (ERFS) and domestic fuel (\hyperlink{https://www.lesechos.fr/industrie-services/energie-environnement/le-chauffage-au-fioul-devient-de-plus-en-plus-cher-147372}{CEREN}).}
% \end{table}

% \end{multicols}
% \renewcommand{\arraystretch}{0.73}


% \begin{table}[ht]
% \centering
% \caption{Positioning towards Yellow Vests, per category}
% {\fontsize{10}{16}\selectfont
% \begin{tabular}{rccccc}
%   \hline \hline
%  & Opposed & Understands & Supports & Is part & PNR \\ 
%   \hline
%   Extreme-left (2\%) & 6\% & 26\% & 51\% & 12\% & 5\% \\ 
%   Left (20\%) & 17\% & 36\% & 36\% & 5\% & 7\% \\ 
%   Center (13\%) & 49\% & 30\% & 15\% & 2\% & 6\% \\ 
%   Right (16\%) & 40\% & 32\% & 20\% & 3\% & 6\% \\ 
%   Extreme-right (9\%) & 11\% & 28\% & 47\% & 10\% & 5\% \\
%   Indeterminate (40\%) & 19\% & 32\% & 30\% & 4\% & 13\% \\
%   \hline
%   Liberal (5\%) & 48\% & 26\% & 18\% & 2\% & 6\% \\
%   Conservative (2\%) & 22\% & 28\% & 30\% & 10\% & 11\% \\
%   Humanist (11\%) & 21\% & 35\% & 29\% & 5\% & 10\% \\
%   Patriot (8\%) & 21\% & 27\% & 39\% & 7\% & 6\% \\
%   Apolitical (21\%) & 21\% & 31\% & 32\% & 4\% & 12\% \\
%   Ecologist (15\%) & 17\% & 39\% & 27\% & 5\% & 12\% \\
%   \hline
%   Rural (21\%) & 20\% & 31\% & 34\% & 6\% & 9\% \\ 
%   <20k (17\%) & 24\% & 28\% & 34\% & 6\% & 9\% \\ 
%   20-100k (14\%) & 22\% & 33\% & 32\% & 4\% & 9\% \\ 
%   >100k (31\%) & 29\% & 34\% & 26\% & 3\% & 8\% \\ 
%   Paris (17\%) & 28\% & 33\% & 25\% & 4\% & 11\% \\
%   \hline
%   No diploma or \textit{Brevet} (30\%) & 21\% & 29\% & 34\% & 5\% & 10\% \\ 
%   \textit{CAP} or \textit{BEP} (24\%) & 23\% & 28\% & 36\% & 6\% & 7\% \\ 
%   \textit{Baccalauréat} (17\%) & 22\% & 35\% & 29\% & 4\% & 11\% \\ 
%   Higher (29\%) & 32\% & 8\% & 36\% & 21\% & 3\% \\
%   \hline
%   Age: 18--24 (12\%) & 23\% & 34\% & 27\% & 4\% & 12\% \\ 
%   Age: 25--34 (15\%) & 21\% & 33\% & 28\% & 7\% & 11\% \\ 
%   Age: 35--49 (24\%) & 25\% & 32\% & 29\% & 5\% & 9\% \\ 
%   Age: 50--64 (24\%) & 21\% & 32\% & 36\% & 4\% & 7\% \\ 
%   Age: $\geq$ 65 (25\%) & 32\% & 30\% & 28\% & 3\% & 7\% \\
%   \hline
%   Income decile: 1 & 25\% & 33\% & 26\% & 3\% & 14\% \\ 
%   Income decile: 2 & 18\% & 31\% & 35\% & 5\% & 11\% \\
%   Income decile: 3 & 17\% & 31\% & 32\% & 7\% & 12\% \\
%   Income decile: 4 & 15\% & 33\% & 37\% & 6\% & 9\% \\
%   Income decile: 5 & 21\% & 29\% & 36\% & 5\% & 8\% \\
%   Income decile: 6 & 26\% & 33\% & 29\% & 6\% & 7\% \\
%   Income decile: 7 & 25\% & 36\% & 28\% & 4\% & 7\% \\
%   Income decile: 8 & 31\% & 31\% & 28\% & 3\% & 8\% \\
%   Income decile: 9 & 39\% & 32\% & 20\% & 3\% & 6\% \\
%   Income decile: 10 & 47\% & 29\% & 15\% & 3\% & 6\% \\
%   \hline
%   Female (52\%) & 21\% & 34\% & 29\% & 5\% & 12\% \\
%   Male (48\%) & 29\% & 30\% & 31\% & 5\% & 6\% \\
%   \hline
%   \textit{Average} & \textit{25\%} & \textit{32\%} & \textit{30\%} & \textit{5\%} & \textit{9\%} \\ 
%   \hline \hline
% \end{tabular}
% }
% \\ $\quad$ \\
% {\footnotesize \textsc{Note:} The percentages in parenthesis express the weighted share of each category from our sample.}
% \label{tab:gilets_jaunes_agglo}
% \end{table}
% % Suggestion : on ne donne pas la part pour le décile de revenu en supposant que c'est 10\% à chaque fois



% \section{Determinants of attitudes}

% \subsection{Control variables\label{app:covariates}}

% Our regression Tables \ref{tab:determinants_attitudes_CC} and \ref{tab:politiques_env} display only the most relevant variables, but --- when specified --- the following additional covariates are included as controls:

% \subparagraph{Socio-demographics:} \textit{respondent's income; household's income; employment status \textnormal{(9 categories)}; socio-professional category \textnormal{(8 categories)}; region of France \textnormal{(10 categories)}; household size; number of people above 14; number of adults; single; number of c.u.; smokes; favored medium for news \textnormal{(5 categories)}.}

% \subparagraph{Political orientation:} \textit{conservative; liberal; humanist; patriot; apolitical.}

% \subparagraph{Energy and exposure to policies:} \textit{heating energy: gaz; heating energy: domestic fuel; surface of accomodation; annual distance travelled by car; fuel economy; type of fuel: diesel; type of fuel: gasoline; number of vehicles; simulated net gain from Tax \& dividend; opinion on public transports; mode of commuting transport.}

% \subsection{Measures for relative preferences\label{app:measures}}

% We constructed the two indexes of section \ref{sec:determinants_attitudes_policies} using the following measures:

% \subparagraph{Norms:} \textit{insulation standards;  pollution standards; roadworthiness standards; prohibition of polluting vehicles.}

% \subparagraph{Taxes:} \textit{kerosene; red meat; urban tolls; climate fund.}

% \subparagraph{Earmarking:} \textit{renovation; renewables; non polluting transport.}

% \subparagraph{Transfers:} \textit{to bottom half; to all; to constrained households.}

\section{Test different wording for winners and losers}


\begin{table}[!h] \centering 
  \caption{Effect of defining winners/losers in terms of purchasing power} 
  \label{table:purchasing_power} 
\makebox[\textwidth][c]{ \begin{tabular}{@{\extracolsep{5pt}}lcccc} 
\\[-1.8ex]\hline 
\hline \\[-1.8ex] 
 & \multicolumn{4}{c}{\textit{Dependent variable:}} \\ 
\cline{2-5} 
\\[-1.2ex] & Poors expected & City dwellers expected & Rich expected & Rural expected \\
& to win & to win & to lose & to lose
\\
 & (1) & (2) & (3) & (4) \\ 
\hline \\[-1.8ex] 
 Constant & 0.058$^{***}$ & 0.207$^{***}$ & 0.009$^{***}$ & 0.352$^{***}$ \\ 
  & (0.007) & (0.010) & (0.003) & (0.012) \\ 
  In purchasing power & 0.045$^{***}$ & $-$0.029$^{**}$ & 0.015$^{***}$ & $-$0.014 \\ 
  & (0.010) & (0.014) & (0.005) & (0.017) \\ 
 \hline \\[-1.8ex] 
Observations & 3,002 & 3,002 & 3,002 & 3,002 \\ 
R$^{2}$ & 0.007 & 0.001 & 0.003 & 0.0002 \\ 
\hline 
\hline \\[-1.8ex] 
& \multicolumn{4}{r}{$^{*}$p$<$0.1; $^{**}$p$<$0.05; $^{***}$p$<$0.01} \\ 
\end{tabular} 
} \end{table} 


\clearpage

\section{Additional specifications for determinants of attitudes}



\begin{table*}[!htbp] \centering 
  \caption{Determinants of attitudes towards diesel taxation} 
  \label{tab:determinants_diesel} 
\makebox[\textwidth][c]{ \begin{tabular}{@{\extracolsep{5pt}}lcccc} 
\\[-1.8ex]\hline 
\hline \\[-1.8ex] 
\\[-1.8ex] & \multicolumn{3}{c}{Acceptance increase in diesel taxation} \\ 
\\[-1.8ex] & (1) & (2) & (3) & (4)\\ 
\hline \\[-1.8ex] 
 Knowledge on CC & 0.045$^{***}$ &  &  &  \\ 
  & (0.008) &  &  &  \\ 
  Ecologist & 0.083$^{***}$ &  &  &  \\ 
  & (0.023) &  &  &  \\ 
  Yellow Vests: PNR & $-$0.041 &  & $-$0.068$^{**}$ &  \\ 
  & (0.030) &  & (0.034) &  \\ 
  Yellow Vests: understands & $-$0.099$^{***}$ &  & $-$0.134$^{***}$ &  \\ 
  & (0.021) &  & (0.023) &  \\ 
  Yellow Vests: supports & $-$0.188$^{***}$ &  & $-$0.289$^{***}$ &  \\ 
  & (0.022) &  & (0.024) &  \\ 
  Yellow Vests: is part & $-$0.162$^{***}$ &  & $-$0.300$^{***}$ &  \\ 
  & (0.040) &  & (0.045) &  \\ 
  Left-right: Extreme-left & 0.083 &  &  & 0.076 \\ 
  & (0.052) &  &  & (0.060) \\ 
  Left-right: Left & 0.033 &  &  & 0.025 \\ 
  & (0.025) &  &  & (0.024) \\ 
  Left-right: Center & 0.016 &  &  & 0.081$^{***}$ \\ 
  & (0.028) &  &  & (0.029) \\ 
  Left-right: Right & $-$0.045$^{*}$ &  &  & $-$0.060$^{**}$ \\ 
  & (0.027) &  &  & (0.026) \\ 
  Left-right: Extreme-right & $-$0.030 &  &  & $-$0.180$^{***}$ \\ 
  & (0.031) &  &  & (0.033) \\ 
  Size of town: -20k & $-$0.001 & 0.002 &  &  \\ 
  & (0.025) & (0.025) &  &  \\ 
  Size of town: 20-100k & 0.013 & 0.016 &  &  \\ 
  & (0.027) & (0.027) &  &  \\ 
  Size of town: +100k & 0.069$^{***}$ & 0.106$^{***}$ &  &  \\ 
  & (0.025) & (0.022) &  &  \\ 
  Size of town: Paris & 0.084$^{**}$ & 0.143$^{***}$ &  &  \\ 
  & (0.041) & (0.026) &  &  \\ 
  Diesel & $-$0.371$^{***}$ & $-$0.474$^{***}$ &  &  \\ 
  & (0.023) & (0.016) &  &  \\ 
  Gasoline & 0.152$^{***}$ &  &  &  \\ 
  & (0.022) &  &  &  \\ 
  Number vehicles & $-$0.022 &  &  &  \\ 
  & (0.019) &  &  &  \\ 
  Frequency of public transit & 0.001 &  &  &  \\ 
  & (0.007) &  &  &  \\ 
 \hline \\[-1.8ex] 
Additional covariates & \checkmark &  &  &  \\ 
Observations & 3,002 & 3,002 & 3,002 & 3,002 \\ 
R$^{2}$ & 0.357 & 0.271 & 0.054 & 0.018 \\ 
\hline 
\hline \\[-1.8ex] 
& \multicolumn{4}{r}{$^{*}$p$<$0.1; $^{**}$p$<$0.05; $^{***}$p$<$0.01} \\ 
\end{tabular} 
} \\ \quad \\ {\footnotesize \textsc{Note:} Standard errors are reported in parentheses. Omitted variables are \textit{Yellow Vests: opposes}, \textit{Age : 18 -- 24} and \textit{Left-right: Indeterminate}. Additional covariates are defined in Appendix C.}
\end{table*} 

\begin{table}[!h] \centering 
  \caption{Effect of being treated on acceptance of shale gas exploitation} 
  \label{table:shale_gas} 
\makebox[\textwidth][c]{ \begin{tabular}{@{\extracolsep{5pt}}lccc} 
\\[-1.8ex]\hline 
\hline \\[.0ex] 
 & \multicolumn{3}{c}{\textit{Dependent variable: Shale gas exploitation: not ``No''}} \\[.5ex] \cline{2-4} \\[-.5ex]
 & (1) & (2) & (3) \\
 \\[-1.8ex] & \textit{OLS} & \textit{OLS} & \textit{logistic} \\ 
\hline \\[-1.0ex] 
 District concerned & $-$0.039$^{**}$ & $-$0.054$^{**}$ & $-$0.059$^{**}$ \\ 
  & (0.019) & (0.024) & (0.024) \\ 
 \hline \\[-1.0ex] 
Controls: Socio-demographics, scores &  & \checkmark & \checkmark \\ 
Observations & 2,847 & 2,847 & 2,847 \\ 
R$^{2}$ & 0.001 & 0.047 &  \\ 
\hline 
\hline \\[-1.8ex] 
& \multicolumn{3}{r}{$^{*}$p$<$0.1; $^{**}$p$<$0.05; $^{***}$p$<$0.01} \\ 
\end{tabular} 
} \end{table} 
% DONE Treated -> District concerned

\begin{table*}[!h] \centering 
  \caption{Determinants of attitudes towards climate policies, additional specifications} 
  \label{tab:politiques_env} 
\makebox[\textwidth][c]{ \begin{tabular}{@{\extracolsep{5pt}}lccc} 
\\[-1.8ex]\hline 
\hline \\[-1.8ex] 
\\[-1.8ex] & \multicolumn{2}{c}{Share of policies} & Tax \& dividend \\ 
\\[-1.8ex] & (1) & (2) & (3)\\ 
\hline \\[-1.8ex] 
 Knowledge on CC & 0.056$^{***}$ &  &  \\ 
  & (0.005) &  &  \\ 
  CC is disastrous & 0.091$^{***}$ &  &  \\ 
  & (0.010) &  &  \\ 
  Diploma (1 to 4) & 0.006 &  &  \\ 
  & (0.004) &  &  \\ 
  Age: 25 -- 34 & $-$0.039$^{**}$ &  &  \\ 
  & (0.018) &  &  \\ 
  Age: 35 -- 49 & $-$0.019 &  &  \\ 
  & (0.017) &  &  \\ 
  Age: 50 -- 64 & 0.005 &  &  \\ 
  & (0.017) &  &  \\ 
  Age: $\geq$ 65 & 0.045$^{**}$ &  &  \\ 
  & (0.018) &  &  \\ 
  Income (k\euro{}/month) & 0.004 &  &  \\ 
  & (0.002) &  &  \\ 
  Sex: Male & $-$0.007 &  &  \\ 
  & (0.009) &  &  \\ 
  Size of town (1 to 5) & 0.008$^{**}$ &  &  \\ 
  & (0.004) &  &  \\ 
  Frequency of public transit & 0.017$^{***}$ &  &  \\ 
  & (0.004) &  &  \\ 
  Left-right: Extreme-left &  & 0.072$^{**}$ & $-$0.065 \\ 
  &  & (0.033) & (0.057) \\ 
  Left-right: Left &  & 0.040$^{***}$ & 0.031 \\ 
  &  & (0.013) & (0.022) \\ 
  Left-right: Center &  & 0.071$^{***}$ & 0.090$^{***}$ \\ 
  &  & (0.015) & (0.026) \\ 
  Left-right: Right &  & 0.029$^{**}$ & $-$0.037 \\ 
  &  & (0.014) & (0.024) \\ 
  Left-right: Extreme-right &  & $-$0.061$^{***}$ & $-$0.155$^{***}$ \\ 
  &  & (0.018) & (0.031) \\ 
 \hline \\[-1.8ex] 
Observations & 3,002 & 3,002 & 3,002 \\ 
R$^{2}$ & 0.142 & 0.018 & 0.017 \\ 
\hline 
\hline \\[-1.8ex] 
& \multicolumn{3}{r}{$^{*}$p$<$0.1; $^{**}$p$<$0.05; $^{***}$p$<$0.01} \\ 
\end{tabular} 
} \\ \quad \\ {\footnotesize \textsc{Note:} Standard errors are reported in parentheses. Omitted variables are \textit{Yellow Vests: opposes}, \textit{Age : 18 -- 24} and \textit{Left-right: Indeterminate}. Additional covariates are defined in Appendix C.}
\end{table*}


\clearpage

\section{Logit regressions for determinants}
\vspace{-0.5cm}
\begin{table*}[!h] \centering 
  \caption{Determinants of attitudes towards climate change (CC) with logit regressions} 
  {\fontsize{8}{12} \selectfont
  \label{tab:determinants_attitudes_CC} 
\makebox[\textwidth][c]{ \begin{tabular}{@{\extracolsep{5pt}}lccccc} 
\\[-1.8ex]\hline 
\hline \\[-1.8ex] 
\\[-1.8ex] & \multicolumn{3}{c}{CC is anthropic} & \multicolumn{2}{c}{CC is disastrous} \\ 
\\[-1.8ex] & (1) & (2) & (3) & (4) & (5)\\ 
\hline \\[-1.8ex] 
 Interest in politics (0 to 2) & 0.034$^{***}$ &  &  & 0.045$^{***}$ &  \\ 
  & (0.013) &  &  & (0.014) &  \\ 
  Ecologist & 0.144$^{***}$ &  &  & 0.186$^{***}$ &  \\ 
  & (0.021) &  &  & (0.027) &  \\ 
  Yellow Vests: PNR & $-$0.097$^{***}$ &  &  & $-$0.069$^{**}$ &  \\ 
  & (0.036) &  &  & (0.034) &  \\ 
  Yellow Vests: understands & $-$0.034 &  &  & $-$0.040$^{*}$ &  \\ 
  & (0.023) &  &  & (0.024) &  \\ 
  Yellow Vests: supports & $-$0.101$^{***}$ &  &  & $-$0.052$^{**}$ &  \\ 
  & (0.025) &  &  & (0.025) &  \\ 
  Yellow Vests: is part & $-$0.196$^{***}$ &  &  & $-$0.079$^{*}$ &  \\ 
  & (0.047) &  &  & (0.044) &  \\ 
  Left-right: Extreme-left & 0.121$^{**}$ &  &  & 0.070 &  \\ 
  & (0.047) &  &  & (0.064) &  \\ 
  Left-right: Left & 0.088$^{***}$ &  & $-$0.035 & 0.104$^{***}$ & $-$0.011 \\ 
  & (0.025) &  & (0.071) & (0.030) & (0.066) \\ 
  Left-right: Center & 0.011 &  & $-$0.079 & 0.030 & $-$0.074 \\ 
  & (0.030) &  & (0.083) & (0.032) & (0.075) \\ 
  Left-right: Right & $-$0.031 &  & $-$0.120 & $-$0.029 & $-$0.140 \\ 
  & (0.029) &  & (0.096) & (0.031) & (0.090) \\ 
  Left-right: Extreme-right & $-$0.012 &  & $-$0.118 & 0.023 & $-$0.084 \\ 
  & (0.034) &  & (0.116) & (0.038) & (0.106) \\ 
  Diploma: \textit{CAP} or \textit{BEP} & 0.042$^{**}$ &  & 0.017 & $-$0.022 & $-$0.011 \\ 
  & (0.020) &  & (0.027) & (0.025) & (0.033) \\ 
  Diploma: \textit{Baccalauréat} & 0.063$^{***}$ &  & 0.109$^{***}$ & 0.025 & 0.148$^{***}$ \\ 
  & (0.024) &  & (0.028) & (0.029) & (0.035) \\ 
  Diploma: Higher & 0.093$^{***}$ &  & 0.160$^{***}$ & 0.095$^{***}$ & 0.229$^{***}$ \\ 
  & (0.025) &  & (0.025) & (0.031) & (0.031) \\ 
  Diploma $\times$ Left-right &  &  & $-$0.010 &  & $-$0.004 \\ 
  &  &  & (0.008) &  & (0.009) \\ 
  Age: 25 -- 34 & 0.048 & $-$0.040 &  & 0.018 &  \\ 
  & (0.042) & (0.041) &  & (0.047) &  \\ 
  Age: 35 -- 49 & $-$0.008 & $-$0.113$^{***}$ &  & 0.026 &  \\ 
  & (0.043) & (0.035) &  & (0.045) &  \\ 
  Age: 50 -- 64 & 0.005 & $-$0.116$^{***}$ &  & $-$0.036 &  \\ 
  & (0.045) & (0.034) &  & (0.047) &  \\ 
  Age: $\geq$ 65 & $-$0.095$^{*}$ & $-$0.228$^{***}$ &  & $-$0.087 &  \\ 
  & (0.057) & (0.035) &  & (0.056) &  \\ 
  Income (k\euro{}/month) & $-$0.011 &  &  & $-$0.010 &  \\ 
  & (0.008) &  &  & (0.009) &  \\ 
  Sex: Male & $-$0.024 &  &  & 0.003 &  \\ 
  & (0.018) &  &  & (0.019) &  \\ 
  Size of town (1 to 5) & 0.006 &  &  & 0.007 &  \\ 
  & (0.008) &  &  & (0.008) &  \\ 
  Frequency of public transit & 0.012 &  &  & 0.007 &  \\ 
  & (0.007) &  &  & (0.008) &  \\ 
 \hline \\[-1.8ex] 
Additional covariates & \checkmark &  &  & \checkmark &  \\  &  &  &  &  &  \\ 
Observations & 3,002 & 3,002 & 1,813 & 3,002 & 1,813 \\ 
\hline 
\hline \\[-1.8ex] 
  & \multicolumn{5}{r}{$^{*}$p$<$0.1; $^{**}$p$<$0.05; $^{***}$p$<$0.01} \\ 
\end{tabular} 
}
}{\\ $\quad$ \\                \footnotesize \textsc{Note:} Average marginal effects are reported, with standard errors in parentheses. Interaction term is computed using numeric variables. Omitted modalities are: \textit{Yellow Vests: opposes}, \textit{Left-right: Indeterminate}, \textit{Diploma: Brevet or no diploma}, \textit{Age: 18 -- 24}. Additional covariates are defined in Appendix C. }                \end{table*}  


\begin{table*}[!h] \centering 
  \caption{Determinants of attitudes towards climate policies with logit regressions}
   {\fontsize{10}{12} \selectfont
  \label{tab:politiques_env_logit} 
\makebox[\textwidth][c]{ \begin{tabular}{@{\extracolsep{5pt}}lcccc} 
\\[-1.8ex]\hline 
\hline \\[-1.8ex] 
\\[-1.8ex] & \multicolumn{2}{c}{Tax \& dividend} & Share of policies & Ecological lifestyle \\ 
\\[-1.8ex] & (1) & (2) & (3) & (4)\\ 
\hline \\[-1.8ex] 
 Knowledge on CC & 0.012$^{***}$ & 0.020$^{***}$ & 0.018$^{***}$ & 0.036$^{***}$ \\ 
  & (0.004) & (0.004) & (0.004) & (0.004) \\ 
  CC is disastrous & 0.022 & 0.034$^{*}$ & 0.081$^{***}$ & 0.138$^{***}$ \\ 
  & (0.017) & (0.018) & (0.020) & (0.018) \\ 
  Interest in politics (0 to 2) & $-$0.019 &  & 0.032$^{**}$ & 0.028$^{**}$ \\ 
  & (0.013) &  & (0.014) & (0.013) \\ 
  Ecologist & 0.107$^{***}$ &  & 0.077$^{***}$ & 0.174$^{***}$ \\ 
  & (0.025) &  & (0.028) & (0.023) \\ 
  Yellow Vests: PNR & $-$0.022 &  & $-$0.054 & $-$0.087$^{**}$ \\ 
  & (0.028) &  & (0.036) & (0.034) \\ 
  Yellow Vests: understands & $-$0.117$^{***}$ &  & $-$0.025 & $-$0.009 \\ 
  & (0.018) &  & (0.024) & (0.022) \\ 
  Yellow Vests: supports & $-$0.207$^{***}$ &  & $-$0.050$^{*}$ & $-$0.020 \\ 
  & (0.019) &  & (0.026) & (0.024) \\ 
  Yellow Vests: is part & $-$0.177$^{***}$ &  & $-$0.079$^{*}$ & $-$0.026 \\ 
  & (0.028) &  & (0.047) & (0.042) \\ 
  Left-right: Extreme-left & $-$0.035 &  & 0.022 & 0.085 \\ 
  & (0.055) &  & (0.065) & (0.055) \\ 
  Left-right: Left & 0.070$^{***}$ &  & $-$0.003 & 0.039 \\ 
  & (0.027) &  & (0.030) & (0.027) \\ 
  Left-right: Center & 0.051$^{*}$ &  & 0.013 & 0.096$^{***}$ \\ 
  & (0.029) &  & (0.033) & (0.028) \\ 
  Left-right: Right & $-$0.022 &  & 0.009 & 0.010 \\ 
  & (0.027) &  & (0.032) & (0.028) \\ 
  Left-right: Extreme-right & $-$0.076$^{**}$ &  & $-$0.023 & 0.010 \\ 
  & (0.034) &  & (0.039) & (0.033) \\ 
  Diploma (1 to 4) & $-$0.002 & 0.003 & 0.007 & $-$0.007 \\ 
  & (0.009) & (0.008) & (0.010) & (0.009) \\ 
  Age: 25 -- 34 & $-$0.039 & $-$0.079$^{***}$ & $-$0.024 & 0.032 \\ 
  & (0.038) & (0.028) & (0.048) & (0.042) \\ 
  Age: 35 -- 49 & $-$0.041 & $-$0.068$^{**}$ & $-$0.015 & 0.050 \\ 
  & (0.037) & (0.026) & (0.046) & (0.040) \\ 
  Age: 50 -- 64 & $-$0.044 & $-$0.078$^{***}$ & $-$0.002 & 0.058 \\ 
  & (0.040) & (0.027) & (0.049) & (0.042) \\ 
  Age: $\geq$ 65 & $-$0.066 & $-$0.075$^{***}$ & 0.0003 & 0.015 \\ 
  & (0.046) & (0.028) & (0.058) & (0.051) \\ 
  Income (k\euro{}/month) & 0.0002 & 0.001 & 0.010 & $-$0.005 \\ 
  & (0.008) & (0.004) & (0.009) & (0.009) \\ 
  Sex: Male & $-$0.050$^{***}$ & $-$0.070$^{***}$ & $-$0.027 & $-$0.066$^{***}$ \\ 
  & (0.017) & (0.017) & (0.020) & (0.018) \\ 
  Size of town (1 to 5) & 0.021$^{***}$ & 0.035$^{***}$ & 0.001 & $-$0.005 \\ 
  & (0.008) & (0.007) & (0.009) & (0.008) \\ 
  Frequency of public transit & $-$0.006 & 0.012$^{*}$ & $-$0.003 & 0.023$^{***}$ \\ 
  & (0.007) & (0.006) & (0.008) & (0.008) \\ 
 \hline \\[-1.8ex] 
Additional covariates & \checkmark & & \checkmark  & \checkmark  \\  &  &  &  &  \\ 
Observations & 3,002 & 3,002 & 3,002 & 3,002 \\ 
\hline 
\hline \\[-1.8ex] 
& \multicolumn{4}{r}{$^{*}$p$<$0.1; $^{**}$p$<$0.05; $^{***}$p$<$0.01} \\ 
\end{tabular} 
}
} \\ \quad \\ {\footnotesize \textsc{Note:} Average marginal effects are reported, with standard errors in parentheses. Interaction term is computed using numeric variables. Omitted modalities are: \textit{Yellow Vests: opposes}, \textit{Left-right: Indeterminate}, \textit{Diploma: Brevet or no diploma}, \textit{Age: 18 -- 24}. Additional covariates are defined in Appendix C. }                \end{table*}

\clearpage


\section{Robustness for the absence of cultural cognition effect}

\begin{table*}[!h] \centering 
  \caption{Robustness of the absence interaction on perceived effects between political orientation and knowledge.} 
  \label{tab:robustness_no_interaction} 
\makebox[\textwidth][c]{ \begin{tabular}{@{\extracolsep{5pt}}lccc} 
\\[-1.8ex]\hline 
\hline \\[-1.8ex] 
\\[-1.8ex] & \multicolumn{3}{c}{CC is disastrous} \\ 
\\[-1.8ex] & (1) & (2) & (3)\\ 
\hline \\[-1.8ex] 
 Constant & 0.404$^{***}$ & $-$0.050 & $-$0.113$^{**}$ \\ 
  & (0.035) & (0.093) & (0.049) \\ 
  Yellow Vests: PNR & $-$0.049 &  & $-$0.017 \\ 
  & (0.041) &  & (0.044) \\ 
  Yellow Vests: understands & $-$0.013 &  & 0.006 \\ 
  & (0.034) &  & (0.039) \\ 
  Yellow Vests: supports & $-$0.020 &  & 0.012 \\ 
  & (0.051) &  & (0.064) \\ 
  Yellow Vests: is part & $-$0.049 &  & 0.040 \\ 
  & (0.079) &  & (0.093) \\ 
  Left-right: Left &  & 0.013 &  \\ 
  &  & (0.067) &  \\ 
  Left-right: Center &  & $-$0.035 &  \\ 
  &  & (0.086) &  \\ 
  Left-right: Right &  & $-$0.061 &  \\ 
  &  & (0.107) &  \\ 
  Left-right: Extreme-right &  & 0.024 &  \\ 
  &  & (0.128) &  \\ 
  Diploma: \textit{CAP} or \textit{BEP} & $-$0.029 &  &  \\ 
  & (0.024) &  &  \\ 
  Diploma: \textit{Baccalauréat} & 0.109$^{***}$ &  &  \\ 
  & (0.027) &  &  \\ 
  Diploma: Higher & 0.203$^{***}$ &  &  \\ 
  & (0.024) &  &  \\ 
  Knowledge CC &  & 0.069$^{***}$ & 0.074$^{***}$ \\ 
  &  & (0.004) & (0.003) \\ 
  \hline \\[-1.8ex]
  Diploma $\times$ Yellow Vests & $-$0.001 &  &  \\ 
  & (0.009) &  &  \\ 
  Knowledge CC $\times$ Left-right &  & $-$0.003 &  \\ 
  &  & (0.004) &  \\ 
  Knowledge CC $\times$ Yellow Vests &  &  & $-$0.001 \\ 
  &  &  & (0.004) \\ 
 \hline \\[-1.8ex] 
Observations & 3,002 & 1,813 & 3,002 \\ 
R$^{2}$ & 0.039 & 0.137 & 0.144 \\ 
\hline 
\hline \\[-1.8ex] 
 & \multicolumn{3}{r}{$^{*}$p$<$0.1; $^{**}$p$<$0.05; $^{***}$p$<$0.01} \\ 
\end{tabular} 
}{\\ $\quad$ \\                \footnotesize \textsc{Note:} Standard errors are reported in parentheses. Interaction term is computed using numeric variables. Omitted modalities are: \textit{Yellow Vests: opposes}, \textit{Left-right: Extreme-left}, \textit{Diploma: Brevet or no diploma}. }                \end{table*}  




\end{document}
